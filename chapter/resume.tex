\begin{resume}

\newcommand\tab[1][1cm]{\hspace*{#1}}
%\newcommand{\itab}[1]{\hspace{0em}\rlap{#1}}
%\newcommand{\tab}[1]{\hspace{.2\textwidth}\rlap{#1}}

\noindent
姓名:张水源 \tab 性别:男\tab  出生日期:1992.2.3 \tab  籍贯:安徽\\

\noindent
2013.9 -- 2016.7  \tab  中国科学院大学 \tab 计算技术研究所 \tab  攻读硕士学位

\noindent
2009.9 -- 2013.7   \tab 厦门大学~~~~~~~~~~~~  \tab 软件学院~~~~~~~~~~~~ \tab  获得学士学位

\resumeitem{攻读硕士学位期间发表的论文}
  \begin{enumerate}[{[}1{]}]
 \item  \textbf{Zhang Shuiyuan}, Xiong, J., Hou, J., Zhang, Q., \& Cheng, X. (2015). HANSpeller++: A Unified Framework for Chinese Spelling Correction. In \emph{Proceedings of the Eighth SIGHAN Workshop on Chinese Language Processing} 2015, 38. Association for Computational Linguistics.

 \item  Zhang, Q., \textbf{Zhang Shuiyuan}, Dong, J., Xiong, J., \& Cheng, X. (2015). Automatic Detection of Rumor on Social Network. In \emph{Natural Language Processing and Chinese Computing} (pp. 113-122). Springer International Publishing.

 \item  Xiong, J., Zhang, Q., \textbf{Zhang Shuiyuan}, Hou, J., \& Cheng, X. (2015). HANSpeller: A Unified Framework for Chinese Spelling Correction. \emph{Computational Linguistics and Chinese Language Processing} 2015, 20.

\item  张巧, \textbf{张水源}, 董健, 熊锦华, 程学旗. (2015). 融合流行度和深层特征的社交网络中谣言识别方法. National Conference of Social Media Processing, 2015.

 \item  Guan, F., \textbf{Zhang Shuiyuan}, Liu, C., Yu, X., Liu, Y., \& Cheng, X. (2014). ICTNET at federated web search track 2014. In \emph{The 23rd Text Retrieval Conference (TREC)}.

\end{enumerate}

\resumeitem{攻读硕士学位期间申请的软著与专利}
\begin{enumerate}[{[}1{]}]
 \item 天玑微博监测与演化分析系统(2015SR017230)

 \item 熊锦华, 张巧, 程学旗, \textbf{张水源}, 许洪波, 张国清, 余智华. (2015). 一种基于用户和微博主题的微博流行度预测方法及系统(2015101094759)
 \item 熊锦华, 张巧, 程学旗, \textbf{张水源}, 许洪波, 余智华. (2015). 一种社交网络谣言识别方法及系统(2015104014582)

\end{enumerate}


\clearpage

\resumeitem{攻读硕士学位期间参加的科研项目} % 有就写,没有就删除
\begin{enumerate}[{[}1{]}]
%  \item XX基金项目“共享存储机群系统的研究”(xxxxx),20xx年1月~20xx年12月
%  \item 国家自然科学基金项目“共享存储机群系统中关键技术研究”(xxxxxxx),20xx年1月~20xx年12月
\item 国家自然科学基金面上项目(No.61173064)
\item 国家科技支撑计划课题(No.2012BAH39B04)
\item 国家重点基础研究发展计划(973)项目(No.2014CB340406)
\item 国家高技术研究发展计划(863)项目(No.2014AA015204)
\item 国家科技支撑计划课题(No.2015BAK20B03)
\item 网络数据重点实验室内部课题ADA系统
\end{enumerate}


\resumeitem{攻读硕士学位期间的获奖情况} % 有就写,没有就删除
  \begin{enumerate}[{[}1{]}]
  \item  2016年~~~~中国科学院大学``三好学生''
  \item  2015年~~~~中国科学院大学``三好学生''
  \item   2015年~~~~中国科学院计算技术研究所``优秀志愿者标兵''
  \item   2015年~~~~中国科学院网络数据科学与技术重点实验室``优秀学生''
  \item   2015年~~~~SIGHAN-2015 Chinese Spelling Check Task 评测第一名
  \item   2014年~~~~Trec 2014 Federated Web Search Task 评测两项任务第一名
  \end{enumerate}
\end{resume}
