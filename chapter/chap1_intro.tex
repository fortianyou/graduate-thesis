
\chapter{引言}
\label{chapter:intro}

\section{研究背景}
当今,人们应该庆幸能够出生在20世纪末,赶上了一个新的计算机技术浪潮,并身处在这个环境中。
有赖于计算机互联网技术,计算机存储和计算技术的快速发展,计算机应用已经深入地影响并改变了人类的生活。

互联网信息爆炸不再是什么新鲜的话题了,但是互联网还在一直发展,移动互联网便是它的延伸。
互联网发展给计算技术带来了许多困难与挑战,并且从来不会中断和消逝。
现代计算机技术的发展的动力源于这种持续不断的冲击。
%直面这些问题与挑战,本文以互联网的兴起为大背景,大数据处理技术为基础,展开了对分布式流式主题模型的设计与实现的探索。

美国图灵奖得主,诺贝尔经济学奖获得者赫伯特$\cdot$西蒙(Herbert A.Simon, $1916\sim2001$)曾说过:“在信息时代,最稀缺的资源不在是信息本身,而是对信息的处理能力。”

什么是大数据,迄今没有公认的定义。相较于传统的数据,人们将大数据的特征总结为5个V,
即体量大(volume)、速度快(velocity)、模态多(variety)、难辨识(veracity)和价值大密度低(value)\cite{cxq2014Survey}。
但大数据的主要难点并不在于体量大,真正难以应付的问题来自于其他几种特性。
在这种环境下,人们往往遇到的是非结构化,海量实时的文本、视频、语音数据。
这些数据常常意味着更大的技术挑战和更复杂的计算,以及更长的计算时间。而在商业应用中,数据处理的时间和质量往往意味着利益。

在金融银行业应用、互联网应用和物联网应用\cite{sun2013bdsc}等许多
场景中,数据往往是一个无穷的序列,并且数据的来源各异,格式复杂多样,
数据可能带有一些时序,但是现有系统通常无法保证数据是按照时间被顺序地处理。
这类数据常被称为流式数据。在不同场景下流式数据呈现出不同的特征,但是他们仍然具有鲜明的共性:

(1) 无限性。在流式场景下,数据以元组为单位,以连续的数据流的形式持续地到达计算系统。

(2) 无序性。虽然每个元组会按照某种时间顺序产生,但是,数据并不是严格按照时间次序到达。

(3) 突发性。数据的流速大小、元组特性数量和数据格式无法被提前预知。

(4) 易失性。在流式场景下,累加的数据量非常巨大。通常来说,数据通过简单的计算之后便会被丢弃,而不会被存储。

(5) 实时性。流式数据不仅是实时产生的,而且要求被实时地计算。短时间内无法返回结果的数据,会长时间停留在内存中,最终的影响系统的性能。

社交网络应用便是一个典型的流式数据应用场景。
社交网络以其快速灵活的特点,己经逐渐改变人们交流的方式,用户可以在社交网络上进行沟通、分享、娱乐等活动。
以微博为例,微博是一个拥有庞大的用户基础,月活跃用户数(MAU)达到3亿的社交应用。
相对于传统的社交媒体,微博的时效性更强,影响范围更大,活跃度更高。微博月阅读量超百亿的领域便达到了18个。
由于微博的流行与普及,微博信息数量呈爆炸式的增长,针对微博数据的处理和应用体现出典型的流式数据处理特征。

\section{研究意义}
我们知道社交网络是机器学习技术应用的主要阵地,许多文本分类,检索与推荐算法都在这个应用领域得到了很好的应用。
主题模型作为一项发现文本潜在语义的机器学习方法和文本挖掘技术,是许多文本分类,推荐与检索算法应用的重要组成部分。
许多公司都有自己的大规模主题模型实现,创造了无数的商业价值。

作为一个研究方向,近年来主题模型受到了广大计算机科学家的爱戴,许多主题模型的方法被提出并应用于各个领域。
不仅如此在生物信息,图像识别以及地理信息等其他学科领域主题模型也有相应的应用。

在以往的应用中,主题模型训练数据一般是静态的有限的。然而,随着网络的普及和社交网络的发展,联网的用户越来越多。
根据上文介绍,社交网络等许多应用领域数据的产生正在呈现出于以往不同的特征,包括海量,高速,实时,变动。
这给主题模型的训练带来许多困扰。
虽然主题模型作为一项重要的文本挖掘技术已经非常成熟,但是至今仍然少有人在流式数据上动态训练主题模型。

\begin{table}[!hbp]
\begin{tabular}{|c|c|c|c|c|c|c|}
\hline
模型名称 & 数据集类型 & 学习类型 & 学习方法 & 分布式并行 & 是否高效 & 随时间演变 \\
\hline
LDA & 静态数据集 & 批量学习 & 变分/Gibbs & 否 & 否 & 否 \\
\hline
PLDA++ & 静态数据集 &  批量学习 & Gibbs & 否 & 否 & 否 \\
\hline
SparseLDA &静态数据集&  批量学习 & Gibbs & 是 & 是 & 否 \\
\hline
AliasLDA &静态数据集&  批量学习 & Gibbs & 是 & 是 & 否 \\
\hline
LightLDA &静态数据集&  批量学习 & Gibbs & 是 & 是 & 否 \\
\hline
OLDA & 开放数据集 & 在线学习 & 变分 & 是 & 是 & 否 \\
\hline
DTM & 静态数据集& 批量学习 & 变分 & 是 & 是 & 是 \\
\hline
\end{tabular}
\label{tab:topic-model}
\caption{相关主题模型实现总结}
\end{table}

当谈及流式机器学习的时候,人们往往需要:

(1) 模型具有时效性。比如天气,如果前两天的天气都是晴天40度,那么第三天下雪的可能性就几乎不可能了。

(2) 模型被经常更新。这需要模型对流数据的烟花和结构具有感知。

考虑到流式环境下模型的时效性要求高,模型更新频繁,许多现有的主题模型实现无法满足要求。
现有的比较具有代表性的大规模主题模型实现包括LightLDA\cite{yuan2015lightlda}和PLDA+\cite{Liu:2011:PPL:1961189.1961198},以及Peacock\cite{Peacock}。
这些算法实现快速高效,扩展性也很好。但是,这些算法实现无一不是为静态数据集设计的。
而在流式数据环境下,算法将面临的是开放无限的数据集,持续增长的词表以及不断演变的数据分布。
显然,现有的算法无法克服上面的问题挑战。

解决这个流式数据上的机器学习的挑战常见的途径是在线学习。
区别于批量学习,在线学习每次只会适用一小批次的数据(而不是整个数据集),并且允许数据的分布信息也随着时间产生变化。
这类方法能够渐进地学习到知识更新演化,并且能修正和加强已有的知识,使得更新后的模型能够适应新的数据,而不必重新对全部数据进行迭代学习。
在线学习降低了算法对时间和空间的要求,更适应与流式数据环境。

除此之外,在流式数据环境下主题模型的设计与实现仍然存在一系列挑战:

(1) 流式数据的特点带来的挑战
流式数据集属于开放数据集,无法直接应用批量学习算法;
流式数据意味着对算法的实时性要求高,现有的许多主题模型算法无法达到这种要求。
许多流式数据的分布会随时间发生演变,现有的主题模型大都不支持模型的动态演变。

(2) 主题模型参数矩阵带来的挑战
流式数据的无限性,意味着动态增长的词表,而词表的大小决定了主题模型的参数个数。一方面我们无法提前预知词表的大小,
另一方面我们无法直接为这种无限大的词表维护参数矩阵。
不仅如此,在分布式环境下大规模参数的更新会带来大量的网络延迟与消耗,这是主题模型实现的重要挑战之一。

面对如此复杂的算法和应用场景,对主题模型的研究不仅具有现实意义,而且还具有一定的代表性。比如,大规模流式数据上的机器学习模型和大规模参数模型,不仅仅在主题模型这个算法中会出现,同样还会出现在一些其他机器学习算法的应用中。

\section{本文贡献}

为了解决分布式流式数据环境下,主题模型实现遇到的挑战。
本文总结并分析了分布式流式数据环境下文本数据的特性,以及主题模型呈现出来的一些特征。
并在分析总结结果的基础之上,提出了分布式流式主题模型设计与实现的具体方案。
总体来说本文的主要贡献在于:

(1) 设计了分布式在线流式主题模型

大规模的分布式并行主题模型的实现得到了国内外各大公司的重视,在生产环境中得到了广泛的应用,产生了巨大的价值。
尽管这些算法实现具有高效稳定的特性,这些算法实现并不是针对流式数据设计的。
本文针对分布式流式数据提出了一种新算法,在线流式主题模型。不同于以往的模型,本文的在线流式主题模型不仅具有动态演变的能力,
而且易于分布式采样算法的设计与实现。%,实验证明这两个算法是高效,稳定,收敛的。

(2) 提出了稠密和稀疏并存的混合参数数据结构

在LDA分布式Gibbs采样算法的实现中,参数词汇主题计数$n(w, t)$起到了关键作用。
如何选取和存储参数$n(w, t)$的数据结构,对算法效率的影响巨大。
因为对于超大规模的主题维度,$n(w,t)$中只有少数非零元,也就是说主题模型参数是一个非常稀疏的矩阵。
不仅如此,参数$n(w, t)$是一个与词汇相关的矩阵。在流式环境下,文本数据中不断有新词出现。
这意味着,分布式流式主题模型无法提前预知词汇表的大小,并且直接对主题模型参数进行存取会造成算法存储和时间效率大大降低。
本文提出的混合参数数据结构,有效地解决了上述两个问题,并大大提升了算法的执行效率。

(3) 采用了高效的Metropolis-Hastings算法

Gibbs采样算法是LDA主题模型参数估计一种主要方法。然而朴素的Gibbs采样算法是一种串行的坐标下降算法,不仅难以实现分布式并行,
而且难以优化,采样算法的时间复杂度为$O(K)$,其中$K$模型的主题个数。这种方法不适用于分布式实时高效的大规模主题模型。
本文为了实现算法的高效采样,结合了Metropolis-Hastings和Alias Table技术使得采样复杂度降低到了$O(1)$。
这种采样技术对于分布式流式主题模型的意义重大,不仅提高了采样的效率,同时令采样算法的时间复杂度不再与主题维度大小线性相关,模型得以训练更大规模的参数。
除此之外,本文算法还对优化了采样顺序和Pipeline网络参数更新,充分利用采样过程中的网络带宽,使得算法运行效率得到了成倍的提升。

\section{章节安排}
本文内容一共分为六章,每章内容组织如下:

第一章为引言部分,主要介绍了流式主题模型的相关背景意义和本文的主要贡献。
在本章中,本文首先介绍了流式数据的应用场景和特性。
接下来结合主题模型的应用,阐述了主题模型在流式数据上遇到的困难与挑战,
对于这些问题的研究不仅能够拓展主题模型在流式数据上的应用,而且对其他大规模流式机器学习具有参考意义。
最后介绍了本文的主要贡献。

第二章从两个方面介绍了国内外的相关工作以及研究现状。
第一个方面是关于流式学习的研究现状。流式数据挖掘的途径通常有两种,一种是基于数据的方案,另外一种是基于任务的方案。
另外一个方面是关于主题模型的研究现状。本文首先介绍了主题模型的发展历史和相关应用背景,之后又介绍了一些主题模型工作的变形。
最后介绍了大规模可扩展主题算法的实现和应用。

第三章介绍了流式主题模型算法设计。
本文首先总结和分析了常见的批量和在线主题模型方案。
本文指出流式主题模型不同于批量主题模型的设计,也不能套用简单的在线主题模型的设计,对于流式主题模型的设计需要考虑到流式数据主要特性。
本文结合了分布式和在线算法的设计思路,提出了一种大规模在线的流式主题模型。

第四章介绍了流式主题模型的参数结构。本章首先介绍了数据并行和模型并行对分布式并行算法的重要意义,以及参数服务器在分布式机器学习算法实现中的重要作用。
然后分析了流式数据环境下,词汇的主要特性和词汇对主题模型参数的影响。在此分析的结果之上,本章提出了稠密和稀疏并存的参数数据结构。
最后在本章提出的参数数据结构的基础之上,本文还介绍了参数数据类型的选择,数据结构稀疏性的维护,以及算法的容错机制。

第五章介绍了一种高效采样算法方案。在这一章,本文首先介绍了Metropolis-Hastings采样算法和Alias Table技术的相关背景。
然后介绍了本文提出的高效采样算法。本文的采样算法主要提出了一种新的提议分布,并应用了Metropolis-Hastings和Alias Table技术,
是算法效率有了大幅的提升。
这些采样算法对分布式流式主题模型意义重大,它们不仅能够提升算法采样的效率,而且使得模型可以训练更大规模的参数。

第六章总结了本文的主要贡献与不足。
