%%% Local Variables:
%%% mode: latex
%%% TeX-master: "../main"
%%% End:

\begin{ack}

研究生生活如白驹过隙,转瞬即逝。在论文即将完成之际, 向所有关心和支持我的人致以最诚挚的谢意!


特别感谢我的指导老师许洪波老师与熊锦华老师,您们严谨的治学态度,丰富的专业知识,循循善诱的指导方式以及精益求精的精神,给了我莫大的启迪和帮助。
在3年的研究生生涯期间,两位老师不仅在学习、科研方面对我悉心指导,给予了很多启发性意见,
更是在为人处世、团队合作和修养提升等方面给予很多中肯的建议,使我对自身有了更清楚认识,这些将是我以后生活中的宝贵财富。
在生活上,两位老师时刻激励着我积极进取,并给予我最大的支持与理解,
您们的言传身教使我终身受益。
在此,向您们致以最诚挚的感谢!

感谢母校国科大为我们的学习生活提供了一个无与伦比的学习与生活环境,陪伴我们度过了三年精彩的青春生活。


感谢网络数据实验室的各位老师和员工,特别是程学旗老师、刘悦老师、宋铟老师、崔连军老师,感谢您们提供了良好的学习和工作环境,
使得我们能够顺利开展各项研究和项目工作,不断取得新成果。
感谢您们在研究生生涯里对我在学习上和生活上无微不至的帮助,使我得到锻炼和成长。


感谢实验室的师兄师姐们:张巧、王千博、林海伦、林基远、赵泽亚、陈波,你们优秀的表现是我学习的榜样,也是我前进的动力。
特别是陈波师兄,在我遇到困难的时候总是尽全力帮助我,对我的疑惑做耐心的解答和指导。

感谢吕珊春、杨龙飞、王成勇、李静远,与你们一起在ADA组的时光十分快乐,与你们一起学习和交流的经历令人愉快而难忘。


感谢在研究生三年中认识的所有同学,特别是欧陈庚、常雨骁、郝晓波、邱志杰、江南、闫肃、李健超、刘伟、陈芳蓉、刘春梅。这些友谊是我一生中最美好的回忆。


最后,我要感谢我的父母,你们一直给我最大的支持和鼓励,让我全心投入到学习生活中,你们永远是我最温暖的港湾。


\end{ack}
