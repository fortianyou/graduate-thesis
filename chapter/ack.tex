%%% Local Variables:
%%% mode: latex
%%% TeX-master: "../main"
%%% End:

\begin{ack}
转眼间又是一个三年,计算所的学习时光是一段特殊的回忆。
在这期间我度过了迷茫彷徨,也经历了斗志昂扬。
但是过往即逝,等待我的是新的征程和新的生活。
值此论文即将完成之际,我要向所有关心和支持我的人们致以最真诚的谢意!

特别感谢我的指导老师郭嘉丰老师和徐君老师,您们治学严谨,
循循善诱,精益求精的品质与精神,令我倍感钦佩,见贤思齐。
研究生期间在两位老师指导和帮助下,我度过了许多学习和科研的难关;
您们让我有机会参与开发学科前沿项目,让我体会到了产学结合的魅力,
更是让我在为人处世、团队合作和自我修养等方面都有机会提升。
您们的言传身教使我终身受益,我将永远铭记在心。

我要感谢温文儒雅的晏小辉老师。
您是良师诤友,不仅在工作学习上悉心指导,还在日常生活中给予了我许多建议。
虽然称您为老师,其实我们年龄相差并不大,
您就像大哥哥一样和我们无话不谈。非常庆幸和珍惜与您共事的时光,这段时光里含有我对研究生阶段最美好的回忆。

我要感谢计算所的各位老师和员工,有高屋建瓴的程学旗老师,诲人不倦的刘悦老师,和蔼可亲的宋铟老师,风趣幽默的崔连军老师,
您们无微不至的帮助与支持,让我们能够顺利地开展项目开发与研究,不断取得新成果。
您们就像勤劳的蜜蜂,兢兢业业,无私奉献,用辛勤的工作陪伴着我们成长。

我还要感谢厦门大学软件学院苏劲松教授。苏老师年轻有为,真知灼见。
我始终不会忘记您对我科研兴趣的培养与启蒙,感谢您给予了我无私的帮助和指导。

感谢研究生认识的小伙伴们,你们当中有我最要好的朋友,有我最崇拜的师兄师姐,感谢您们陪我度过了喜怒哀乐。

感谢母校国科大为我们的提供首屈一指的学习生活环境,以及无与伦比的教学资源。

最后,我要感谢我的父母家人,您们无私无尽的爱是我所有动力的源泉。

\begin{flushright}
郭天佑于计算所

2017年5月 
\end{flushright}
\end{ack}
