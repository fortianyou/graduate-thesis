\chapter{流式主题模型算法设计}
\label{chapter:design}
大规模数据上的主题模型在国内外各大公司都得到了实现与应用,产生了重要的影响和价值。
流式数据环境下,主题模型的实现能够学习到主题的演化,并且动态地更新模型。
本章介绍了主题模型主要技术,并且提出了流式主题模型的设计。

\section{数学符号和术语}
本文通篇会使用到“词汇”,“文档”,“语料”,“主题”等术语。
比如,我们有时也会使用“隐变量”来表示用于标记抽象“主题”的变量。
为了避免这种歧义以及其他可能的问题,规范使用这些术语能够帮助描述和理解模型的设计。

如下,我们形式化地定义了一些术语与符号标记:
\begin{itemize}
\item 词汇:表示数据的基本单位,定义存在于一个词汇表中的元素。
\item 文档:表示为一个长度为$N$的词汇序列,$\mathbf{w}=(w_1, w_2, ..., w_N)$,其中$w_n$表示序列中的第$n$个词汇。
\item 语料:表示为一个大小为$M$的文档集合,$\mathbf{C=(w_1, w_2, ..., w_M)}$。
\item 主题:表示在文本预料集合$C$中语义内聚的多项式隐变量$z$。主题变量$z$的个数是一个有用户定义的常数$K$。
\item $\alpha, \eta$:表示Dirichlet模型的先验参数。
\item $\Theta, B$:表示LDA主题模型的后验参数。$\Theta=\mathbf{\{\theta_1, \theta_2, ...,\theta_M\}}$,
其中$\mathbf{\theta_m}$表示文档$\mathbf{w_m}$得到的主题分布。$B = \mathbf{ \{\beta_1, \beta_2, ...,\beta_K\}}$,
其中$\mathbf{\beta_k}$表示主题变量$z = k$的词汇分布。
\end{itemize}

\section{Latent Dirichlet Allocation(LDA)}
LDA的基本思路是将文档表示成主题随机变量的混合模型,每个主题表示为在词汇上的概率分布。
LDA提出的语料$C$文档生成过程描述如下:

0. 初始化参数$B$,$\{\beta_k \sim Dir(\eta_k)~ |~k \in [1, ... K]\}$

1. 从泊松分布中抽样第m篇文档的长度$N_m$,$N_m \sim Poisson(\xi)$

2. 从Dirichlet先验分布中抽样$\mathbf{\theta_m}$, $\mathbf{\theta_m} \sim Dir(\alpha_m)$

3. 抽样文档中的所有$N_m$个词项$w_n$:

~~~(a) 从多项式分布中抽样词项主题分配$z_n$,$z_n \sim Multinomial(\mathbf{\theta_m})$
   
~~~(b) 抽样一个词汇$w_n$,$w_n \sim p(w_n|\beta_{z_n})$

4. 对所有语料$C$中剩余的文档执行1 $\sim $ 3。\\
在上面的生成过程中,隐含着若干个假设:

(1) 给定主题模型的个数为$K$;

(2) 词生成概率矩阵$B$是一个$K \times V$的矩阵,$V$表示词汇表的大小;

(3) 每个文档的主题分布$\theta$都是$K$维的向量;

(4) 使用泊松分布作为文档长度的分布并不是决定的,如果有可以使用更好的分布;

(5) 文档长度$N$和其它主题模型的参数变量$\Theta, B, z$是相互独立的。\\
对于一个$K$维的Dirichlet变量$\mathbf{v}$落在一个$(K-1)$维的单纯形中(如果一个向量中的每个元素都不小于0,且总和为1,则落在$(K-1)$维的单存形中),
并且拥有如下概率密度:
\begin{equation}
Dir(\mathbf{v }| \mathbf{a} ) = \dfrac{\Gamma(\sum_{i=1}^K{a_i})}{\prod_{i=1}^K{\Gamma(a_i)}} 
v_1^{a_1-1} v_k^{a_k-1}
\end{equation}
其中,$\mathbf{a}$是一个$K$维的向量,且所有元素都大于0;$\Gamma(x)$表示Gamma函数。

Dirichlet函数对于单纯形来说是个简单的分布,它属于指数分布族,有有限的充分统计量。
Dirichlet还有一个重要的特性,这个特性便是和多项式分布式共轭。在许多应用中都会使用Dirichlet作为多项式分布的生成模型。

给定参数$\alpha, \eta$,参数$\Theta, B$,主题$\mathbf{z}$,数据$\mathbf{C}$的完全概率为:
\begin{equation}
p(\Theta, B, \mathbf{z, C} | \alpha, \eta) = \prod_{k=1}^K{Dir(\beta_k|\eta)}
\prod_{m=1}^M{\prod_{n=1}^N{Dir(\theta_m|\alpha)p(z_{mn}|\theta_m)p(w_{mn}|\beta_{z_{mn}})}}
\end{equation}
其中$p(z_{mn}|\theta_m)$表示主题取值为$z_mn$时的概率。通过积分,从上式可以得到数据集的概率:
\begin{equation}
p(\mathbf{C} | \alpha, \eta) = \prod_{k}^K{\prod_{m=1}^M{\int{Dir(\beta_k|\eta)} 
\int{Dir(\theta_m|\alpha)\prod_{n=1}^N{\sum_{z_{mn}}{p(z_{mn}|\theta_m)p(w_{mn}|\beta_{z_{mn}})}}}} d\theta_m }d\beta_k
\end{equation}

图\ref{fig:LDA}展示了LDA的概率图模型表示。图中$\theta_d$表示文档主题分布参数,$\beta_k$表示主题$z=k$的生成词概率分布。
$\alpha, \eta$分别表示LDA模型中文档主题分布$\theta_d$和主题生成词概率分布$\beta_k$的Dirichlet先验参数。
$z, w$表示每个文档中被采样生成的主题和词汇。

\begin{figure}[htb]\centering
  \includegraphics[width=0.5\linewidth]{LDA}
  \caption{LDA图模型表示}
  \label{fig:LDA}       % Give a unique label
\end{figure}

\subsection{批量LDA参数估计}
现有的LDA参数估计方法大多属于批量求解方法,主要包括两类算法,即变分方法和Gibbs采样。
前者的代表是Blei等人的VB方法\cite{blei2003latent}以及Teh等人的CVB\cite{teh2006a};
后者主要是Griffiths等人的CGBS\cite{griffiths2004finding}以及它的一些改进和扩展\cite{porteous2008fast, yao2009efficient, li2014reducing}。

在变分方法中,算法并非去直接估计真实的后验分布,而是借助另外一个更为简单的分布$q(z,\theta, \beta)$渐进地逼近正式的后验。
这些参数的最优值是通过最大化如下ELBO(Evidence Lower BOund, ELBO)得到的:
\begin{equation}
\log p(\mathbf{w} | \alpha, \eta) \ge L(\mathbf{w, \phi, \gamma, \lambda}) \triangleq 
\mathbb{E}_q{[\log p(\mathbf{w, z, \theta, \beta} | \alpha, \eta)]} - 
\mathbb{E}_q{[\log q(\mathbf{z, \theta, \beta})]}.
\end{equation}
其中$\phi, \gamma, \lambda$是变分方法引入变分参数,$\gamma$和$\lambda$分别是变分方法中$\theta$和$\beta$的Dirichlet先验。
值得强调的是,这里最大化ELBO等价于最小化分布$q(\mathbf{z, \theta, \beta})$和后验分布$p(\mathbf{z, \theta, \beta} | \mathbf{w}, \alpha, \eta)$之间的KL距离。

\begin{algorithm}[htb]  
\caption{ Batch Variational Bayes for LDA} 
\label{alg:bvb} 
\begin{algorithmic}[1] 
\Require Corpus $\mathbf{C = \{w_1, w_2, ..., w_M\}}$
\State Random initialize parameter $\lambda$
\While {relative improvement in $L(\mathbf{w, \phi, \gamma, \lambda}) > 0.00001$}
\State E step:
\For { m = 1 to M }
\State Set $\gamma_{mk} = 1$
\Repeat
\State Update $\phi_{mwk}, \gamma_{mk} $
\Until{ change in $\gamma_{mk} $ is relative small}
\EndFor
\State M step:
\State Update $\lambda_{kw}$
\EndWhile
\end{algorithmic}  
\end{algorithm}  

根据VB方法,可以得到上面的批量变分参数优化算法。

Gibbs采样算法是另一种主要LDA参数估计方法,并且更易于实现,因而在许多算法实现中得到了应用\cite{Liu:2011:PPL:1961189.1961198, Peacock, li2014scaling}。 

Gibbs采样算法(GiBbs Sampling)是马可夫蒙特卡洛(Markov-chain Monte Carlo, MCMC)算法的一种特例\cite{mackay2002information, hesterberg2012monte}。
MCMC算法的主要思路是构造一个非周期马氏链,并按照某一个转移概率反复地对马氏链的各个状态进行抽样,最终马氏链的状态分布会收敛于一个稳定的分布$\pi$。
虽然,我们无法直接获知$\pi$的具体值,但是我们仍然能够从分布中得到样本。
按照这种定义,只需要从分布中获取得到足够多的样本便可以很快地计算出稳定分布的近似估计$\hat{\pi}$。

\begin{figure}[htb]\centering
  \includegraphics[width=0.8\linewidth]{Gibbs_Sampling}
  \caption{使用Gibbs采样算法学习LDA过程}
  \label{fig:Gibbs_Sampling}       % Give a unique label
\end{figure}

CGBS算法中提出的Gibbs采样算法通过积分消除了模型参数$\theta, \beta$,使得算法相比于直接的GBS算法能够更快地收敛:
\begin{equation}
\label{eq:pzw}
\begin{aligned}
p(\mathbf{z , w | \alpha, \eta}) &= p(\mathbf{w | z , \alpha, \eta}) p(\mathbf{z| \alpha, \eta}) \\
p(\mathbf{w | z, \alpha, \eta})                &= \left( \dfrac{\Gamma(V\eta)}{\Gamma(\eta)^V} \right)^K
\prod_{k=1}^K{\dfrac{ \prod_w^V \Gamma( n_{k,w} + \eta )}{\Gamma( n_{k, \cdot}+ V \eta)}} \\
p(\mathbf{z | \alpha, \eta})                    &=\left( \dfrac{\Gamma(K\alpha)}{\Gamma(\alpha)^K} \right)^M
\prod_{m=1}^M{\dfrac{ \prod_k^K \Gamma( n_{k,m} + \alpha)}{\Gamma( n_{\cdot, m}+ K \alpha)}} \\
\end{aligned}
\end{equation}
其中,$n_{k,w}$表示语料中词汇$w$的主题分配为$z=k$的计数,$n_{k, m}$表示文档中主题分配为$z=k$的词项的计数。
$n_{k, \cdot} = \sum_w^V{n_{k, w}} , n_{\bullet, m} = \sum_k^K{ n_{k, m}}$。

\begin{algorithm}[htb]
\caption{ Batch Collapsed Gibbs Sampling for LDA} 
\label{alg:bcgbs}
\begin{algorithmic}[1] 
\Require Corpus $\mathbf{C = \{w_1, w_2, ..., w_M\}}$
\State Random assign topic to each word in $\mathbf{C}$
\State Collect and summary $n_{k, w}$ and $n_{k, m}$
\While {Not converged or iter > MAX\_ITERATION}
\For { m = 1 to $M$ }
\For { n = 1 to $N_m$}
\State Sample $z_{mn} \sim p(z_{mn} | \mathbf{z}_{\neg (mn)}, \mathbf{w}) $
\State Update  $n_{k, w}$ and $n_{k, m}$ according to $z_{mn}$
\EndFor
\EndFor
\EndWhile
\end{algorithmic}  
\end{algorithm}  

借助式子\ref{eq:pzw},可以得到下面的抽样分布的定义:
\begin{equation}
\begin{aligned}
p( z_i = k | \mathbf{z}_{\neg i},  \mathbf{w}) 
&\propto p( z_i = k , w_i = t | \mathbf{z}_{\neg i}, \mathbf{w}_{\neg i}) \\
& = \dfrac{p(\mathbf{w |z}) p(\mathbf{z})}{p(\mathbf{w}_{\neg i} | \mathbf{z}_{\neg i})p(\mathbf{z}_{\neg i})} \\
& = \dfrac{ n_{k, w_i}^{\neg i,j} + \eta }{ n_{k, \cdot}^{\neg i,j} + V\eta}
\dfrac{n_{k, m}^{\neg i,j} + \alpha}{n_{\cdot, m}^{\neg i,j} + K \alpha}
\end{aligned} 
\end{equation}
其中$n^{\neg i,j}$表示不包括$z_i$的计数。这个抽样分布的式子很直观,
左边其实主题生成词概率的Dirichlet后验估计,右边则是文档主题分布的Dirichlet后验估计,正好反映了LDA模型先抽样主题,然后再抽样根据主题抽样词汇的生成过程。

算法\ref{alg:bcgbs}展示了批量CGBS的算法学习过程。
算法首先随机为语料中的每个词项分配一个主题,并统计$n_{k, w}$和$n_{k, m}$的值。
之后算法按照类似坐标轴下降的方法,顺序地抽样语料中的每个词汇的主题。
每个迭代将会遍历语料中所有的文档的所有词项。若干次迭代之后算法,最终会收敛。
值得注意的是,这边坐标轴下降方法中并不一定需要顺序地抽样语料中的每个词汇,实际上词汇之间先后的抽样顺序并不相关,可以按照任意顺序进行抽样。

批量的算法是LDA模型求解算法的主流算法,在语料大小可以完全被加载时,这种算法具有实现简单更加精确的优点。然而在数据量太大,数据分布不稳定的情况下,在线算法往往体现出来更好的性能。

\subsection{在线LDA参数估计}
\begin{algorithm}[htb]
\caption{Online Variational Bayes for LDA} 
\label{alg:olda}
\begin{algorithmic}[1] 
\State Define $\rho_t \triangleq (\tau_0 + t)^{-\kappa}$
\State Random initialize parameter $\lambda$
\For{ t = 0 to $\infty$}
\State E step:
\State Set $\gamma_{tk} = 1$
\Repeat
\State Update $\phi_{twk}, \gamma_{tk} $
\Until{ change in $\gamma_{tk} $ is relative small}
\State M step:
\State Update $\tilde{\lambda}_{kw}$
\State Set $\mathbf{\lambda}=(1- \rho_t) \mathbf{\lambda} + \rho_t \tilde{\mathbf{\lambda}}$
\EndFor
\end{algorithmic}  
\end{algorithm}  

在标准批量学习方法中,算法每一轮迭代都需要遍历整个数据集,这种计算方法带来的资源消耗非常高。
特别在主题模型应用中,这种现象更为常见——主题模型往往需要借助大规模的数据集上训练,
来挖掘人为无法标记的主题。虽然,使用分布式并行也能解决这种数据规模大的问题,但是这并不妨碍人们提出LDA参数估计的在线模型。

OLDA(Online Latent Dirichlet Allocation, OLDA)\cite{hoffman2010online}提出的在线VB方法将主题模型看作概率矩阵分解,并参考在线矩阵分解的技术\cite{mairal2010online},提出了如下在线算法\ref{alg:olda}。

算法\ref{alg:olda}中,语料集视为无穷大,并且按照时间分片一次一小批次地输入,所有词汇$w$来自于固定的词汇集$V$。
在整个算法执行过程中,$\lambda$起到了重要的作用,在这里$\lambda$参数表示LDA模型$\beta$的变分Dirichlet后验参数。随着时间的推移,$t$时刻之前的$\lambda$并没有被丢弃,而是通过一种衰减的方式保留下来。

衰减的权重设计为$\rho_t \triangleq (\tau_0 + t ) ^{-\kappa}$,这个式子中$\kappa \in (0.5, 1], \tau \ge 0$。
这么做的好处就是不断会有新的后验知识加入进来,并且旧的先验知识和后验知识的比例关系得到协调(时间越靠前的权重越小),使得模型在对全局的知识都有很好的掌握,,并很快地收敛。

On-Line LDA\cite{alsumait2008on-line}算法则从Gibbs采样算法的角度出发设计了在线LDA算法。

\begin{algorithm}[htb]
\caption{Online Gibbs Sampling for LDA} 
\label{alg:on-linelda}
\begin{algorithmic}[1]
\Require $ a, b, \omega^{\sigma}, S^t, t \in [1, +\infty)$
\For{ t = 1 to $\infty$}
\If{t == 1}
\State $\eta_k^t = b, k \in \{1, ..., K\}$
\Else
\State $\eta_k^t = B_k^{t-1} \omega^{\sigma}, k \in \{1, ..., K\}$
\EndIf
\State Random assign topic to each word in $S^t$
\For{ each word $w$ in $S^t$ }
\State GibbsSampling($S^t, \eta^{t - 1}, \alpha$)
\EndFor
\State Collect and summary $n_{k,w}^{(t)}$和$n_{k}^{(t)}$
\State Set $\beta_{k,w}^{(t)} = n_{k,w}^{(t)}$
\State Set $B_k^t = B_k^{t - 1} \cup \beta_k^{(t)}$
\EndFor
\end{algorithmic}  
\end{algorithm}  

算法\ref{alg:on-linelda}中,$\omega$是一个维度为$\sigma \time 1 $的向量,表示了时刻$t$之前的$\sigma$时刻内的$\beta$在先验中所占的权重。$\beta_k^t$是在$t$时刻词汇与主题$k$共现计数,是一个维度为$V^t \times 1$的向量。
$B_k^t$是一个$V^t \times \sigma$的演变矩阵。

通过这种窗口化的机制,算法不仅能够快速收敛,还能使得模型更聚焦于最近的数据。
对于那些时间太久远的潜在语义则有可能被遗忘,因而能够更精确地捕获主题在时间轴上的演变。
On-Line LDA借此将主题模型很好地应用于主题检测与跟踪。

\section{流式主题模型算法} 
流式数据带有显著的特性:无限性、无序性、突发性、易失性和实时性。
互联网社交网络和自媒体的广泛应用和发展,使得文本数据的增长速度越来越快。
并且网络上热议的话题经常因为各种现实世界的环境或者突发因素而改变(比如季节、社会事件等等),并且带有强烈的话题时效性。
\begin{figure}[htb]\centering
  \includegraphics[width=1\linewidth]{movie-hot}
  \caption{2016年12月在映电影微博热议度变化趋势}
  \label{fig:LDA}       % Give a unique label
\end{figure}

对于此类数据,不仅内容演变快,同一时刻到达的数据量也非常可观。
大多数据公司都会采用分布式流式处理引擎来保证高吞吐率,高速率的数据处理。
常见的分布式流式计算引擎包括Storm,Spark Streaming和Samza。

可见海量的数据,频繁演变的主题趋势是一个很现实的问题。
如果仍然使用批量的算法来训练主题模型,那么在应用中不得不频繁地重新训练模型。
显然这种方法代价高昂,并且未必能够很好地利用所有的数据以及捕获主题的演变。
在线算法可以对主题演变进行良好的建模,但是目前的算法并没有提出大规模的分布式流式数据上的在线主题模型算法。

考虑到P.Domingos和G.Hulten\cite{Domingos01catchingup}提出
的数据流算法应该具有的特性:

(1) 对每个数据样本只需要很少的运算时间

(2) 使用内存大小固定,与处理的样本数据总量无关

(3) 在构建模型的过程中,所有训练数据只会被计算一遍或少数几遍

(4) 随时产生独立于样本顺序的模型

(5) 具有概念迁移能力。

在线流式主题模型的提出主要是受到了OLDA\cite{hoffman2010online}和On-Line LDA\cite{alsumait2008on-line}以及DTM\cite{blei2006dynamic, wang2012continuous}等论文的启发。

\begin{equation}
\label{eq:dir-gonge}
Dir(\mathbf{p} | \eta + \beta ) = Dir(\mathbf{p} | \eta ) + MultiCount( \beta )
\end{equation}

根据Dirichlet-Multinomial共轭\ref{eq:dir-gonge},$Dir(\mathbf{p} | \eta )$表示数据的先验分布,
$MultiCount(\beta)$表示观测到数据多项式统计,$Dir(\mathbf{p} | \eta)$表示给定先验分布与观测数据得到的相应后验分布。

\begin{figure}[htb]\centering
  \includegraphics[width=0.5\linewidth]{graph-ostm}
  \caption{在线流式主题模型图模型表示}
  \label{fig:OSTM}       % Give a unique label
\end{figure}

如图\ref{fig:OSTM},该模型以流式计算为背景,将流式数据按照批次分割成不同的时间片。
仅有相邻的时间片模型之间产生关联。
图中$\beta$对应于$MultiCount(\beta)$多项式统计;图展示的是$t$时刻之后,算法将会得到一个后验的模型后验分布$Dir(\mathbf{p} | \eta + \beta_t)$。
为了使得$t$时刻训练得到的模型对$t+1$时刻的模型产生影响,本文的算法将$t$时刻的后验分布$Dir(\mathbf{p} | \eta + \beta_t )$作为
$t+1$时刻模型的先验分布。为了使得模型具有迁移演化能力,我们引入了$\lambda \in (0, 1)$表示衰减权重,应用于上一个时刻的观测数据之上。
最终在$t+1$时刻的模型先验分布可以表示为$Dir(\mathbf{p} | \eta + \lambda \beta_t )$。

本文的方法虽然与以往的一些工作有相似之处,但是存在明显的区别。
其中最重要的区别便是本文模型针对流式数据而提出,最终目标是方便实现大规模在线分布式并行采样算法的实现。
而DTM, OLDA以及On-Line LDA等模型并非针对流式数据的模型,因而无法适应在流式数据上的进行模型训练。

除此之外,DTM与OLDA均采用变分推理的方法来训练模型,而本文的方法使用了采样的方法,更有利于分布式并行实现。
DTM虽然对时间进行了建模,当时DTM是一个批量的学习算法,无法适应流式数据场景。
OLDA虽然采用了在线的学习算法,但是该算法并没有对模型的动态演变能力进行建模。

On-Line LDA与本文的模型一样都具有演变能力,同时也是基于采样算法实现。
但是On-Line LDA的模型先验分布表示为$Dir(\mathbf{p} | \eta + \sum_{i=1}^{win}{\beta_{t-i}} )$,
需要保存窗口$win$以内的所有历史数据,不仅对于拥有海量参数的大规模模型是无法容忍的,还会造成参数维护困难。
另外,本文提出的分布式MH采样算法效率要远远高于On-Line LDA采用的单机串行Gibbs采样算法。

\begin{algorithm}[]
\caption{Online Stream Topic Model}
\label{alg:onlineStreamLDA}
\textbf{\underline{Task Scheduler:}}
\begin{algorithmic}[1]
\Function{UPDATE\_VOCAB\&PARAMS}{$t$}
\State Collect and summary vocabulary $V^t$ at $S^t$
\State Add unkown words to $V_g$, $V_g = V_g \cup V^t$
\State Set $\beta^t(w, k) = \eta$, push $\beta^t(w, k)$, for $w \in (V_g - V^t)$, and update $\beta^t(k)$ accordingly.
\State Synchronize and pull parameter set $\beta^t(k), \beta^t(w, k)$ from servers
\State Set $\bigtriangleup \beta^t(k) = \lambda_t \beta^t(k), \bigtriangleup \beta^t(w, k) = \lambda_t \beta^t(w, k)$
\State Push $\bigtriangleup \beta^t(k), \bigtriangleup \beta^t(w, k)$  to servers and synchronize.
\EndFunction
\State Create a global vocabulary $V_g$
\For{ t = 1 to $\infty$}
\State Issue UPDATE\_VOCAB\&PARAMS(t)
\State Issue WORKER\_SAMPLE($t$) to all workers
\EndFor
\end{algorithmic}
\textbf{\underline{Workers $r = 1, ..., R$:}}
\begin{algorithmic}[1]
\Function{ LOAD\_DATA}{$t$}
\State Load a part of training data of $S^t$ as $C^t$
\State Pull the parameter set $\beta^t(k)$ and $\beta^t(w, k)$ from servers
\EndFunction
\Function{ WORKER\_SAMPLE}{$t$}
\State Issue LOAD\_DATA($t$)
\State Initialize topic assignment to each word in $C^t$ with prior knowledge.
\For{ each word $w$ in $C^t$ }
\State DO\_SAMPLING($C^t, \beta^t(k), \beta^t(w, k), \alpha$)
\EndFor
\State Collect and summary $n_{k,w}^{(t)}$ and $n_{k}^{(t)}$
\State Set $\bigtriangleup \beta^t(k) = n_{k}^{(t)}, \bigtriangleup \beta^t(w, k) = n_{k,w}^{(t)}$ and push $\bigtriangleup \beta^t(k), \bigtriangleup \beta^t(w, k)$ to servers
\EndFunction
\end{algorithmic}  
\textbf{\underline{Servers:}}
\begin{algorithmic}[1]
\Function{SERVER\_ITERATE}{$t$}
\State Aggregate $\bigtriangleup \beta^t(k) = \sum_r^R{\bigtriangleup \beta^t_r(k)}, \bigtriangleup \beta^t(w, k) = \sum_r^R{\bigtriangleup \beta^t_r(w, k)}$
\State $\beta^{(t+1)}(k) =\beta^t(k) +  \bigtriangleup \beta^t(k), \beta^{(t+1)}(w, k) = \beta^t(w, k) + \bigtriangleup \beta^t(w, k)$
\EndFunction
\end{algorithmic}
\end{algorithm}  

论文\cite{smola2010an}中提出了一种分布式并行Gibbs采样算法。在该分布式算法中,数据被均衡地分布式存储在不同的位置,
因而并行算法可以通过数据并行的方式实现。考虑到:
\begin{equation}
p( z_i = k | \mathbf{z}_{\neg i},  \mathbf{w}) 
 \propto \dfrac{ n_{k, w_i}^{\neg i,j} + \eta }{ n_{k, \cdot}^{\neg i,j} + V\eta}
(n_{k, m}^{\neg i,j} + \alpha)
\label{eq:sample-prob}
\end{equation}
主题抽样的概率只会依赖于$n_{k, w_i}^{\neg i,j}, n_{k, \cdot}^{\neg i,j}, n_{k, m}^{\neg i,j}$,因而与其他文档的主题计数无关。
主题模型并行采样算法的核心思想便是少数文档的采样并不会对全局主题分布$n(k)$和词汇-主题表$n(w, k)$造成很大的影响。
这意味着并行算法可以放松对$n(k)$和$n(w, k)$的同步(严格的同步极大的降低分布式算法的效率),从而可以加大算法的并行度同时保证算法收敛。
这种分布式模型并行思路被许多高效的主题模型分布式并行实现所采纳。

在本文的算法实现中我们延续了论文\cite{smola2010an}的思想,设置$n(k)$和$n(w, k)$为全局参数,
并假设在某些时刻这两个参数会得到同步(比如,以时间片为粒度)。

算法\ref{alg:onlineStreamLDA}总共分为三个部分Task Scheduler, Workers, Servers。
其中Task Scheduler是一个单机的算法调度模块,Workers是分布式并行的主要进程,Servers表示参数服务器。
$\lambda_t \in (0, 1)$表示衰减权重,可以有不同的定义。
这么做使得$\beta^t(k)$和$\beta^t(w, k)$拥有衰减的历史后验信息。
根据Dirichlet和多项式分布共轭的属性,我们将已观测到$\beta^t(w, k)$信息作为主题生成词多项式分布的Dirichlet先验。

流式主题模型算法迭代以时间为单位,在每个Worker的采样算法开始之前,各个Worker得到的模型参数是同步的。
而在每个时间片内部,所有的Worker之间是完全异步的,Worker进程之间的采样过程不需要互相等待。
另外在流式环境下,不断会有新词出现,模型必须能够动态的维护新词,本文将在后面的章节中讨论如何维护动态词表。

\section{实验分析}

\subsection{实验设置}
本文所有实验均采用如下设置,

\em{数据源:} 本文的实验数据均来自于某未公开动态实时数据,该数据中包含财经,体育,娱乐等多种类型内容;

\em{计算引擎:} 在本文的所有实验中,均采用了Spark Streaming作为流式计算引擎,拥有3台计算节点和1个参数服务器节点;

\em{节点信息:} Centos 6.6, 128G内存, 24 Intel(R) Xeon(R) CPU E5-2620 0 @ 2.00GHz。

\subsection{效果分析}
\begin{figure}[htb]\centering
  \includegraphics[width=1\linewidth]{exp-stream-perplexity}
  \caption{在线流式主题模型运行效果图}
  \label{fig:exp-stream-perplexity}       % Give a unique label
\end{figure}

上图\ref{fig:exp-stream-perplexity}展示的是在线流式主题模型在某一数据上的运行效果图。
图中评价指标为每个时间片数据上的Perplexity。
实验中我们设置了衰减权重主题模型个数$K = 100$,每个时间片10000个文档,每个时间片算法迭代30轮。
实验中我们共有4组实验,分别设置了$\lambda = 0, 0.5, 0.25, 0.125$。
不难发现,每个时间片上的主题模型训练都呈现出收敛的趋势,且衰减权重的设置对在线流式主题模型的效果影响重大。

当$\lambda = 0$时,也就是不考虑历史数据的影响,在线流式主题模型相当于重新训练主题模型;
当$\lambda \ne 0 $时,模型将历史数据作为先验,并且$\lambda$值越大模型受到先验的影响越大,收敛越快,但是Perplexity相对较大。
实际上,在许多机器学习的研究中都提到过先验知识能够提升模型的泛化能力。
先验知识太强也有可能使得模型在训练集上的效果变差。本文的实验正好符合这个现象。
在许多应用中,先验参数的选择是一个经验问题。特别在本文的在线应用场景中,我们很难确定什么参数最合适。
但这并不意味着我们无法选择一个合适的参数。比如,我们可以将在线实时的反馈结果作为选择参数的依据。

尽管我们并不知道最合适的参数是多少,但是我们发现本文的算法在不同的时间片上都具有收敛的态势。
只要参数设置合理,对于一批新数据模型能够迅速地收敛到一个比较好的位置,说明本文的算法具有适应新数据的能力。

\section{本章小结}
本章主要介绍了大规模流式主题模型的设计。首先介绍了LDA模型的基本理论背景,包括LDA模型的介绍,以及常见LDA参数估计方法。
从算法框架角度看,又有多种不同的批量估计方法和在估计方法。
在流式数据环境下,数据呈现出海量、高速、实时等特性。这些特性使得批量算法和简单的在线算法无法被直接应用。
因此,本文提出了一种大规模流式主题模型的设计方案。
该模型不同以往的主题模型,它是针对流式数据特性而设计的,便于高效的分布式采样算法的实现。
在后续的章节中,本文将更加详细地介绍如何高效地实现这种流式主题模型算法。
