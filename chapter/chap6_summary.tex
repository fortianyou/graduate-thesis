\chapter{总结与展望}
\label{chapter:summary}
\section{本文总结}

主题模型是一种用以表达文本语义的常用模型,其在社交网络和信息检索,广告推荐等各个相关领域都有重要的应用。
主题模型具有参数规模大,训练数据规模大等特性,当训练语料和参数规模足够大时,主题模型能够很好地表达文本的语义,
因为大规模主题模型训练受到了国内外许多互联网科技公司的青睐。

随着计算机互联网技术的应用和发展,现今的数据生长趋势越来越迅猛,特别是社交网络应用中。
类似于微博,twitter等应用是文本数据产生的主要阵地。研究报告表明,这类社交网络中每天会有上亿条数据产生,并且呈逐年增长的趋势。
这些高速,海量实时的流式数据带来了新的技术难题,特别是在此类数据上的机器学习应用。
流式数据是一个新的应用场景,在流式数据上训练主题模型具有重要意义,比如社交网络上用户感兴趣的话题通常会随时间发生演变。

流式数据挖掘的作为一个研究课题被研究者常年关注。
根据研究者对流式数据的经验总结,流式数据具有无限,实时,易失,无序,突发等特性。
不仅如此,在社交网络中数据的分布随着时间的推移会发生迁移和演变。
许多研究都说明流式学习系统应该具有的特性包括:
(1) 对每个数据样本只需要很少的运算时间;
(2) 算法那使用的内存大小固定,不随处理数据增长而增长;
(3) 模型能够被实时动态地更新;
(4) 模型具有演变和概念迁移能力。
传统批量学习算法显然不符合流式学习系统的特殊要求。

但是流式数据环境下主题模型的训练存在一系列挑战:
(1) 来自于海量参数的分布式存储和并行更新同步的挑战;
(2) 来自于流式学习对于算法实时性要求高的挑战;
(3) 来自于流式数据环境下动态增长的词表的挑战。

为了应对上述的挑战,本文主要工作在于设计与实现高效的流式主题模型。
本文从三个角度出发,解决了流式主题模型的主要难题。
首先,本文提出了一种高效实时的具有演化能力的在线流式主题模型。
它能够有效地克服流式数据的无限性,不仅算法能够动态地更新模型,而且具有概念迁移的能力。
其次,本文从算法实现的角度出发设计了稠密和稀疏并存的混合参数数据结构,并在此基础上设计了参数的备份与恢复机制,保证了算法高可用性。
最后,本文又采用了Metropolis-Hastings和Alias Table技术,使得每次采样的复杂度降低到了$O(1)$,满足了流式学习对算法实时性的要求。


\section{下一步工作}
本文的工作主要集中在对流式主题模型的设计与实现。
虽然本文提出的算法设计与实现能够有效地应对流式数据环境下的一系列挑战,
但是在实现和实验过程中我们仍然发现了一些其他的问题和待提升空间,以及一些其它相关的研究内容。

(1) 在线模型的深入研究与改进 

本毕业设计的工作内容主要集中在如何高效地实现稳定高可用的分布式流式主题模型。
虽然在本工作中提出在线流式主题模型的设计,但是本文中并没有深入对在线流式主题模型进行更深层次的研究。
实际上,模型的设计也是一个具有重要意义的工作,相信还有比本文更好的模型,期待后期能够继续改进。

(2) 参数数据结构可以进一步稀疏化 

在实验过程中,我们发现随着主题模型的训练和演变,词汇主题分布会不断地发生变化并且逐渐稀疏化,
这不仅会出现在 Sparse(V, Z) 参数表中还会出现在 Dense(V, Z) 表 中。因此利用 Dense(V, Z) 的稀疏特性,有望进一步提升算法的运行效率。 

(3) 其他流式机器学习算法的研究 

本文研究工作遇到的挑战,不仅出现主题模型中,同样还会出现其他 nlp,信息检 索以及其他领域的其他算法中。
将本文的研究成果迁移到其他算法的应用之中也是一个 值得尝试的工作。

